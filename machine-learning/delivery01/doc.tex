\documentclass[11pt,letterpaper]{article}
\bibliographystyle{amsplain}

\usepackage[utf8x]{inputenc}
\usepackage{ucs}
\usepackage[spanish]{babel}
\usepackage{amsmath}
\usepackage{amsthm}
\usepackage{amsfonts}
\usepackage{amssymb}
\usepackage{url}
\usepackage{pstricks}
\usepackage[affil-it]{authblk}

\author{J. Mauricio Mej{\'i}a Castro\thanks{\tt jesidm.mejiac@utadeo.edu.co}}
\title{Taller de Repaso de Matemáticas}
\affil{Universidad Jorge Tadeo Lozano}

\newcommand{\dpartial}[2]{\frac{\partial#1}{\partial#2}}
\newcommand{\upla}[2]{(#1_1,\ #1_2,\ \ldots,\ #1_{#2})}

\theoremstyle{plain}
	\newtheorem{prop}{Proposición}[section]
	\newtheorem{teor}[prop]{Teorema}
	\newtheorem{corol}[prop]{Corolario}
	\newtheorem{lema}[prop]{Lema}

\theoremstyle{definition}
	\newtheorem{defi}{Definición}[section]
	\newtheorem{ejem}{Ejemplo}
	\newtheorem{ejer}{Ejercicio}

\theoremstyle{remark}
	\newtheorem{nota}{Nota}[section]
	\newtheorem{notac}{Notación}

\begin{document}

\maketitle
%\tableofcontents
\newpage

\section{Álgebra Lineal}
	\subsection{Producto Punto y Norma}
\begin{defi}[Producto punto] Sean %
$\mathbf{x},\ \mathbf{y} \in \mathbb{R}^{n}$ $n$-vectores tales que %
$\mathbf{x} = \upla{x}{n}$ y $\mathbf{y} = \upla{y}{n}$. %
El {\bf producto punto}, o {\bf producto escalar} de $\mathbf{x}$ y %
$\mathbf{y}$ es el número
	\begin{equation}
	\mathbf{x}\cdot \mathbf{y} = x_1y_1+\cdots +x_ny_n.
	\end{equation}
\par La {\bf norma}, {\bf longitud} o {\bf magnitud} de $\mathbf{x}$ es la raíz %
cuadrada positiva
	\begin{equation}
	\left\| \mathbf{x} \right\| = \sqrt{ \mathbf{x} \cdot \mathbf{x} } %
	= \left( x^2_1 + \cdots + x^2_n \right)^{1/2}.
	\end{equation}
\par 
Observe que la norma siempre está definida dentro de los números reales, pues $\left\|\mathbf{u}\right\|^2$ %
siempre es no negativo.
\par También la {\bf distancia euclidiana} entre $\mathbf{x}$ y $\mathbf{y}$ %
se define como 
	\begin{equation}
	d  \left\| \mathbf{x} - \mathbf{y} \right\|.
	\end{equation}
\end{defi}



\end{document}